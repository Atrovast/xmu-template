%!TEX program = xelatex
\documentclass{xmu}
\begin{document}

% 基础信息

% \print % 电子版 / 打印版(打印版将在某些偶数页产生空白页)
% \design % 毕业设计 / 毕业论文(取消注释即为毕业设计)
% \minor % 主修 / 辅修(取消注释即为辅修)
\title{中文标题}{English Title}
\author{你的姓名}
\idn{你的学号}
\college{你的学院}
\subject{你的专业}
\grade{你的年级}
\teacher{校内指导教师\; 职称}
\otherteacher{校外指导教师\; 职称} % 注释则不显示校外指导教师
\pubdate{完成时间}
\keywords{中文关键词}{English keywords}

% 封面、承诺书

\maketitle

% 如果想要将致谢放在最前,可将最后的致谢移动至此

% 中文摘要

\begin{abstract}
    摘要内容。
\end{abstract}

% 英文摘要

\begin{enabstract}
    Abstract Contents.
\end{enabstract}

% 目录

\tableofcontents

% 正文

\xmuchapter{一级标题(章)}{Chapter}
\xmusection{二级标题(节)}{Section}
\xmusubsection{三级标题(小节)}{Subsection}
\subsubsection{四级标题} % 四级标题不显示在目录中,无需英文
正文内容,脚注\footnote{脚注内容},引用参考文献\cite{cite1}。

% 参考文献

\begin{reference}
    % 如果需要使用 bib 文件导入参考文献,则取消注释下一行
    % \bibliography{references.bib}
    % 已自动按照 GB/T 7714-2005 设置参考文献的引用格式

    % 手动添加参考文献
    \begin{thebibliography}{1} % 括号内数字为参考文献条数
         \bibitem[1]{cite1} 参考文献1
    \end{thebibliography}
\end{reference}

% 附录

\begin{appendix}
    附录内容。
\end{appendix}

% 致谢(默认在最后)

\begin{acknowledgement}
    致谢内容。
\end{acknowledgement}

\end{document}